\documentclass{article}

\usepackage{palatino}

\usepackage{amsmath}
\usepackage{amsfonts}
\usepackage[utf8]{inputenc}
\usepackage[colorlinks=true]{hyperref}

% macros
\newcommand{\LL}{\mathcal{L}}
\newcommand{\half}{\frac{1}{2}}
\newcommand{\norm}[1]{\left|\left|#1\right|\right|}
\newcommand{\dd}{d}
\newcommand{\ddd}[2]{\frac{\partial #1}{\partial #2}}
\newcommand{\block}[1]{\left(#1\right)}
\newcommand{\mat}[1]{ \begin{pmatrix} #1 \end{pmatrix} }
\newcommand{\inv}[1]{#1^{-1}}
\newcommand{\argmin}[1]{\underset{#1}{\textrm{argmin}}}


\newcommand{\p}[1]{#1^{+}}
\newcommand{\m}[1]{#1^{-}}
\newcommand{\s}[1]{#1^{*}}
\newcommand{\xp}{\p{x}}
\newcommand{\xm}{\m{x}}
\newcommand{\vp}{\p{v}}
\newcommand{\vm}{\m{v}}
\newcommand{\vs}{\s{v}}
\newcommand{\fp}{\p{f}}
\newcommand{\fm}{\m{f}}
\newcommand{\fs}{\s{f}}
\newcommand{\Dx}{\Delta x}
\newcommand{\Dv}{\Delta v}


\begin{document}
\title{Newton Solver Reference}
\author{Fran\c{c}ois Faure and Matthieu Nesme}
\date{\today}
\maketitle
%
\begin{abstract}
  Detailing the non-linear time-stepping scheme implemented in the CompliantNLImplicitSolver component.
\end{abstract}
%


\section{Notations}


\begin{itemize}
\item $x$: positions
\item $v$: velocities
\item $f(x,v,t)$: forces for given positions and velocities at time $t$
\item $h$: time step
\item $\m{(.)}$, $\p{(.)}$: a state at, respectively, the beginning and the end of the time step
\item $\Dx=\xp-\xm$: variation of position during the time step
\item $\Dv=\vp-\vm$: variation of velocity during the time step
\item $\alpha$, $\beta$: blending parameters such as $\fs=\alpha\fp+(1-\alpha)\fm$ and $\vs=\beta\vp+(1-\beta)\vm$. Corresponding Data are called implicitVelocity and implicitPosition.
\item $M$: mass
\end{itemize}

\section{Euler Integration}

\[
   \left \{
   \begin{array}{r c l}
      \Dx  & = & h\vs \\
      M\Dv & = & h\fs \\
   \end{array}
   \right .
\]

from explicit $\alpha=\beta=0$ to implicit $\alpha=\beta=1$.


\section{Non-linear Solver}

The method computes the next velocity $\vp$, such that $e \equiv M\Dv-h\fs$ is satisfied.\\


Based on the Newton's method, an approximate solution is iteratively improved by solving a linear equation system based on the Jacobian of the residual of the equation to satisfy.
A first guess is computed with the regular, linearized system (cf the linear time-stepping scheme in the Compliant plugin doc). \\


Stating that \[ e \equiv M(\vp-\vm)-h(\alpha\fp+(1-\alpha)\fm)\] we obtain the jacobian \[ \ddd{e}{\vp}=M-h\alpha\ddd{\fp}{\vp}\]

A first order approximation of the Taylor serie of $\fp=f(\xp,\vp,t+h)$ gives

\[
   \begin{array}{r c l}
      \fp  & = & f(\xm,\vm,t)+\ddd{f}{x}\Dx+\ddd{f}{v}\Dv \\
           & = & \fm+K\Dx+B\Dv \\
           & = & \fm+K(h(\beta\vp+(1-\beta)\vm))+B(\vp-\vm)
   \end{array}
\]

with $K=\ddd{f}{x}$ the stiffness matrix and $B=\ddd{f}{v}$ the damping matrix. \\


So \[ \ddd{\fp}{\vp}=h\beta K+B\] and \[\ddd{e}{\vp}=M-h\alpha B-h^2\alpha\beta K\]


\section{Newton Step Length}

Two different strategies are implemented:
\begin{itemize}
\item naïve sub-step approach: a predefinied portion (Data 0$<$newtonStepLength$<$1) of the correction is applied successively while the error is decreasing. This is simple and quite efficient. For 10\% (default) it seems to always converge.
\item Backtracking algorithm (Data newtonStepLength=1): try to find the maximum amount of correction to apply that decreased "sufficiently" the error. The line search described in \textit{Numerical Recipies} (chapter Globally Convergent Methods for Nonlinear Systems of Equations) is employed. Even if this method should be better, it sometimes diverges, maybe because of its magical thresholds?!
\end{itemize}


\end{document}

