% ---- graphiques
\usepackage[pdftex]{graphicx}
\usepackage{wrapfig}
\usepackage{color}
\usepackage{pst-tree}
%\usepackage{hyperref}

% for accents
\usepackage[latin1]{inputenc}
\usepackage[T1]{fontenc}

\usepackage{algorithm}
\usepackage{algorithmic}

\definecolor{darkgreen}{rgb}{0,0.4,0}
\definecolor{darkblue}{rgb}{0,0,0.4}
\definecolor{darkgray}{rgb}{0.2,0.2,0.2}

% ---- inclusion de codes
\usepackage{listings}
\lstset{showstringspaces=false,tabsize=4,basicstyle=\scriptsize\sffamily,breaklines=true,breakatwhitespace=true,framexleftmargin=5mm, frame=shadowbox, framesep=1pt,rulesepcolor=\color{darkgray},rulesep=.5pt,keywordstyle=\bf\color{blue},commentstyle=\color{magenta},stringstyle=\color{red},numbers=left,numberstyle=\tiny,numbersep=5pt,columns=flexible}

\lstdefinestyle{bash}{language=bash}
\lstdefinestyle{Perl}{language=Perl}
\lstdefinestyle{Python}{language=Python}
\lstdefinestyle{C++}{language=C++,emph={__global__,__shared__,__syncthreads,blockIdx,threadIdx,float3,float4},emphstyle=\bf\color{darkgreen}}
\lstdefinestyle{DTD}{language=XML}
\lstdefinestyle{XML}{language=XML,usekeywordsintag=false,markfirstintag=true}
%begin{latexonly}
\newcommand{\includecode}[2]{
\lstinputlisting[style=#1]{#2}
}
%end{latexonly}


%\lstnewenvironment{code}{}{}
\lstnewenvironment{code_bash}{\lstset{style=bash}}{}
\lstnewenvironment{code_perl}{\lstset{style=Perl}}{}
\lstnewenvironment{code_python}{\lstset{style=Python}}{}
\lstnewenvironment{code_cpp}{\lstset{style=C++}}{}
\lstnewenvironment{code_dtd}{\lstset{style=DTD}}{}
\lstnewenvironment{code_xml}{\lstset{style=XML}}{}

\newcommand{\textcode}[1]{{\sf #1}}




\newcommand{\sofa}{SOFA}
\newcommand{\todo}[1]{}
\newcommand{\eg}{\textit{e.g.} }

\renewcommand{\vec}[1]{\ensuremath{\mathbf{#1 }}} % vector
\newcommand{\Vx}{\vec{x} } % position vector
\newcommand{\Vv}{\vec{v} } % velocity vector
\newcommand{\Va}{\vec{a} } % acceleration vector
\newcommand{\Vf}{\vec{f}} % force
\newcommand{\Vdv}{\vec{\delta\Vv}} % change of velocity vector (unknown in implicit CG, and used in constraint solver
\renewcommand{\P}{\mat{P} } % projection to a constrained space.

\newcommand{\JNL}{\mathbf{\mathcal{J}} }     % mapping des positions
\newcommand{\J}{\mat J }                 % mapping lineaire
\newcommand{\M}{\mat M }             % matrice de masse
\newcommand{\K}{\mat K }             % matrice de raideur
\newcommand{\B}{\mat B }             % matrice d'amortissement
\newcommand{\G}{\mat G }             % jacobien des contraintes



% ---- inclusion de codes
\definecolor{darkgreen}{rgb}{0,0.4,0}
\definecolor{darkblue}{rgb}{0,0,0.4}
\definecolor{darkgray}{rgb}{0.2,0.2,0.2}


% macros mathematiques
\newcommand{\ma}[1]{\ensuremath{\mathbf {#1}}}
\newcommand{\ve}[1]{\ensuremath{\mathbf {#1}}}

\usepackage{amsmath}
\usepackage{amsfonts}
\usepackage{amssymb}

% character styles
\newcommand{\bm}[1]{\ensuremath{\mathbf{{#1}}}}
\newcommand{\mcal}[1]{\mbox{$\mathcal #1$}} % rondes math
\newcommand{\bmcal}[1]{\mbox{\boldmath $\mathcal #1$}} % rondes grasses math
\newcommand{\ensemble}[1]{\mbox{$\mathbb{#1}$}}
\newcommand{\RRR}{\mbox{$\ensemble{R}^3$}} 


