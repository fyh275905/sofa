\documentclass[12pt]{article}
\usepackage{setspace}
\usepackage{amsmath}
\usepackage{titling}

\setlength{\droptitle}{-10em}   % This is your set screw
\title{ODE solvers recursive equations}
\author{Aurelie Degletagne}


\begin{document}
\begin{doublespace}

\maketitle
\tableofcontents

\section{Notations}
In each following equation h is the time step, $x_{k}$,$v_{k}$,$a_{k}$   the position, velocity and acceleration at the time step k, K the stiffness, M the mass, $r_{m}$ the rayleigh mass, $r_{k}$ the rayleigh stiffness and $F_{ext}$ the external forces.

\section{Euler solver}

\subsection{EulerExplicit solver}
It is the simplest time integration solver.
\\The integration scheme is based on the following equations:
\\$ v_{k+1} = v_{k}+ha_{k} $
\\$ x_{k+1} = x_{k}+hv_{k} $

\subsection{EulerImplicit solver}
It is a semi-implicit time integrator using backward Euler scheme for first and second degree Ordinary Differential Equations (ODE). 
This is based on [Baraff and Witkin, Large Steps in Cloth Simulation, SIGGRAPH 1998]
\\ The integration scheme is based on the following equations:
\\$ v_{k+1} = v_{k}+h\frac{Fext_{k}+v_{k}(-(r_{k}+h)K-r_{m}M)}{h(h+r_{k})K+(1+hr_{m})M} $
\\$ x_{k+1} = x_{k}+hv_{k} $

\section{Newmark solver}
It is an implicit time integrator using Newmark scheme.
The implementation computes $a_{k}$ directly then solves the first equation to compute $a_{k+1}$, and finally computes the new velocity $v_{k+1}$ and the new position $x_{k+1}$. In Sofa one uses for Newmark coefficients $\gamma=0.5$ and $\beta=0.25$.
\\$ a_{k+1} = \frac{Fext_{k} + h*cstAcc*a_{k}+cstVel*v_{k}}{(h^2\beta + h\gamma r_{k})K+(1+h\gamma r_{m})M} $
$\left\{
\begin{array}{r c l}
cstAcc &=& (-r_{m}(1-\gamma)M-h(0.5-\beta)K-rk(1-\gamma)K)\\
cstVel &=& (-r_{m}M-(h+r_{k})K)
\end{array}
\right.$
\\$ v_{k+1} = v_{k}+h(1-\gamma)a_{k}+h\gamma a_{k+1} $
\\$ x_{k+1} = x_{k}+hv_{k}+0.5h^2(1-2\beta)a_{k}+h^2\beta a_{k+1} $

\section{VariationalSymplectic solver}
It is explicit and implicit time integrator using the Variational Symplectic Integrator as defined in: Kharevych, L et al. “Geometric, Variational Integrators for Computer Animation.” ACM SIGGRAPH Symposium on Computer Animation 4 (2006): 43–51. p is the momentum.

\subsection{VariationalSymplecticExplicit solver}

$ v_{k+1} = \frac{2(Fext_{k}+p_{k}-hr_{m}v_{k}M)}{M} $
\\$ p_{k+1} = Mv_{k+1} $
\\$ x_{k+1} = x_{k}+hv_{k+1} $

\subsection{VariationalSymplecticImplicit solver}
The current implementation for implicit integration assume $alpha=0.5$ (quadratic accuracy) and uses several Newton steps to estimate velocity.
\\$ v_{k+1} = \frac{2(hFext_{k}+2p_{k})}{4M+h^2K+4h(r_{m}M+r_{k}K)} $
\\$ p_{k+1} = (M+\frac{h^2}{4})v_{k+1}+\frac{h}{2}Fext_{k} $
\\$ x_{k+1} = x_{k}+hv_{k+1} $

\section{RungeKutta}
It is a popular explicit time integration method, much more precise than euler explicit solver.

\subsection{Runge Kutta second order}
It is the Runge-Kutta method with order 2 or the middle point rule. Functions are evaluated two times at each step.
\\ \underline {At time t + h/2}
\\$ x_{k+\frac{1}{2}} = x_{k}+\frac{1}{2}hv_{k}$
\\$ v_{k+\frac{1}{2}} = v_{k}+\frac{1}{2}ha_{k} $
\\$ a_{k+\frac{1}{2}} = \frac{Fext_{k+\frac{1}{2}}}{m}$
\\ \underline {At time t+h}
\\$x_{k+1} = x_{k}+h*v_{k+\frac{1}{2}}$
\\$v_{k+1} = v_{k}+h*a_{k+\frac{1}{2}}$
\\$a_{k+1} = \frac{Fext_{k+1}}{m}$

\subsection{Runge Kutta fourth order}
It is the Runge-Kutta method with order four. Functions are evaluated four times at each step.
\\  \underline{Variables}
\\$stepBy2 = \frac{h}{2}$  $       $ $stepBy3 = \frac{h}{3}$ $       $  $stepBy6 = \frac{h}{6}$
\\ \underline{At time t :}
\begin{itemize}
\item \textbf{Step 1}
\\$k1v = v_{k}$
\\$k1a = a_{k}$
\item \textbf{Step 2}
\\$k2x = x_{k} + k1v*stepBy2$
\\$k2v  = v_{k} + k1a*stepBy2$
\\$k2a  = (Fext_{stepBy2}(k2x,k2v))/m$
\item \textbf{Step 3}
\\$k3x  = x_{k} + k2v * stepBy2$
\\$k3v  = v_{k} + k2a * stepBy2$
\\$k3a  = (Fext_{stepBy2}(k3x,k3v))/m$
\item \textbf{Step 4}
\\ $ k4x  = x_{k} + k3v * h $
\\ $ k4v  = v_{k} + k3a * h $
\\ $ k4a  = (Fext_{k+1}(k4x,k4v))/m $
\end{itemize}
 \underline{At time t + h :}
\\$x_{k+1} = x_{k} + k1v * stepBy6 + k2v * stepBy3 + k3v * stepBy3 + k4v * stepBy6$
\\$v_{k+1} = v_{k} + k1a * stepBy6 + k2a * stepBy3 + k3a * stepBy3 + k4a * stepBy6$
\\$a_{k+1} = (Fext{k+1})/m$

\section{CentralDifference solver}
It is an explicit time integrator using central difference (also known as Verlet or Leap-frog). The equations are separated in two cases: either the rayleigh mass is null or not. 
\begin{itemize}
\item \underline {if $rm=0$}
\\$a_{k+1} = \frac{Fext_{k+1}}{m}$
\\$v_{k+1} = v_{k}+ha_{k+1}$
\\$x_{k+1} = x_{k}+hv_{k+1}$
\item \underline{if $rm!=0$}
\\$a_{k+1} = \frac{Fext_{k+1}}{m}$
\\$v_{k+1} = cstVel * v_{k} + cstAcc*a_{k}$      
$\left\{
\begin{array}{r c l}
cstVel &=&  \cfrac{\cfrac{1}{h}-\cfrac{r_{m}}{2}}{\cfrac{1}{h}+\cfrac{r_{m}}{2}}\\
cstAcc &=& \cfrac{1}{\cfrac{1}{h}+\cfrac{r_{m}}{2}}
\end{array}
\right.$
\\$x_{k+1} = x_{k}+hv_{k+1}$
\end{itemize}
\end{doublespace}
\end{document}